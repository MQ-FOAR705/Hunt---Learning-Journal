\documentclass{article}
\usepackage[utf8]{inputenc}
\usepackage{ragged2e}
\usepackage{hyperref}

\title{\textbf{LEARNING JOURNAL}}
\author{Emily Hunt 44888619}
\date{August 2019}

\begin{document}
\maketitle
\begin{FlushLeft}

To view on Overleaf \url{https://www.Overleaf.com/read/mbyzzhcjkmtv}

\tableofcontents
\pagebreak

\section{Overleaf}

\subsection{Objective: Create a Section}
\textbf{Time Stamp} 10:52am 12 August\\
\textbf{Actions:} In the source tab enter
\verb|\section{section title goes here}|\\
\textbf{Results:} The numbered section was created successfully.\\
\textbf{Errors:} None.\\
\textbf{Solutions/Notes:} For unnumbered sections, include an asterisk before the \verb|{section title}|. For a subsection, replace \verb|\section| with \verb|\subsection|.

\subsection{Objective: Bold Some Text}
\textbf{Time Stamp:} 11:00am  12 August\\
\textbf{Actions:} In the source tab enter \verb|\textbf{text you wish to bold goes here}|.\\
\textbf{Results:} The text was put into bold successfully.\\
\textbf{Errors:} None.\\
\textbf{Solutions/Notes:} To underline text replace \verb|\textbf| with \verb|\underline|, and use \verb|\textit| for italics.

\subsection{Objective: Put Text on a New Line}\label{sec:underfull}
\textbf{Time Stamp:} 11:10am  12 August\\
\textbf{Actions:} Insert \verb|\\| at the end of the line I want the space at.\\
\textbf{Results:} The text was placed on a new line.\\
\textbf{Errors:} At first using the two back slashes prompted the error - Underfull /hbox (badness 10000) in paragraph at lines 15-19.\\
\textbf{Solutions/Notes:} I removed the \verb|\\| and instead inserted an extra space between the lines in the source tab using the enter key. This seemed to work successfully and removed the error. I then went back later and tried again with the two back slashes and this no longer produced an error so I reinserted them where I wanted the line breaks.

\subsection{Objective: Align Text to the Left}\label{sec:logcommand}
\textbf{Time Stamp:} 11:20am 12 August\\
\textbf{Actions:} I inserted \verb|\usepackage[document]{ragged2e}| at the top of the document.\\
\textbf{Results:} This aligned the entire text the left.\\
\textbf{Errors:} I received a note in the log the command had changed. \\
\textbf{Solutions/Notes:} I read further through the page \url{https://www.Overleaf.com/learn/latex/Text_alignment} where I got the initial command from, and checked some of the example documents. I put \verb|\usepackage{ragged2e}| at the top of the document and placed \verb|\begin{FlushLeft}| at the start of the text I wanted to align and \verb|\end{FlushLeft}| at the end of the document, and this seemed to work successfully and remove the error.

\subsection{Objective: Include Code in the Text}
\textbf{Time Stamp:} 11:35am 12 August\\
\textbf{Actions:} I wanted to allow some parts of the code to be seen in the text so I could properly note my actions, and just including them would execute the command. I started by using the listings package, so included\\ \verb|\usepackage{listings}| at the top of the document, and put \verb|\begin{lstlisting}| before the code I wanted to be visisble and \verb|\end{lstlisting}| after.\\
\textbf{Results:} This worked successfully.\\
\textbf{Errors:} However, while not an error, it placed the code on a new line and in a larger font which made the document difficult to read.\\
\textbf{Solutions/Notes:} I then tried using \verb|\begin{verbatim}| and \verb|\end{verbatim}| around the code I wanted, but this had a similar look to using the listings package. Then, I tried the \verb|\verb| code, and this had the effect I desired.

\subsection{Objective: Create an Bullet List}
\textbf{Time Stamp:} 3:00pm 12 August\\
\textbf{Actions:} I put \verb|\begin{itemize}| at the beginning of the list I wanted to create, \verb|\item| for every bullet point, and \verb|\end{itemize}| at the end of the list. \\
\textbf{Results:} This worked successfully.\\
\textbf{Errors:} No errors.\\
\textbf{Solutions/Notes:} For an ordered list change \verb|{itemize}| to \verb|{enumerate}|

\subsection{Objective: Sync my Learning Journal on Overleaf to GitHub}
\textbf{Time Stamp:} 8:00pm 12 August\\
\textbf{Actions:} I went to the Menu tab, clicked on GitHub under sync, clicked 'Link to GitHub', signed into my GitHub account and authorised Overleaf.\\
\textbf{Results:} My GitHub and Overleaf accounts were linked.\\
\textbf{Errors:} Going back the menu and clicking GitHub, it says the document is not linked to a GitHub repository, and that I need to create one. I would like to link it to my existing repository created in class.\\ 
\textbf{Solutions/Notes:} I need to work out how to put it into my existing repository. (See \autoref{sec:submodule})

\subsection{Objective: Insert a Clickable Link}
\textbf{Time Stamp:} 8:10pm 12 August\\
\textbf{Actions:} Add \verb|\usepackage{hyperref}| to the preamble. To insert the link, include in the source, \verb|\url{link goes here}|.\\
\textbf{Results:} A link was successfully created.\\
\textbf{Errors:} None\\
\textbf{Solutions/Notes:}

\subsection{Objective: Download a .tex File}
\textbf{Time Stamp:} 12:00pm 13 August\\
\textbf{Actions:} Go to the Menu at the top left hand corner of the page, and under download, click Source. This will download it as a zipped file. Unzip the file after downloading to access the .tex file.\\
\textbf{Results:} File was downloaded and opened successfully.\\
\textbf{Errors:} None\\
\textbf{Solutions/Notes:}

\subsection{Objective: Put Text On a New Page}
\textbf{Time Stamp:} 9:30pm 13 August\\
\textbf{Actions:} Insert \verb|\pagebreak| in the source code before the line I want to appear on a new page .\\
\textbf{Results:} The text appeared on the next page as intended.\\
\textbf{Errors:} None\\
\textbf{Solutions/Notes:}

\subsection{Objective: Insert a Break Between Lines}
\textbf{Time Stamp:} 9:30pm 19 August\\
\textbf{Actions:} Insert \verb|\vspace{5mm}| where I want the break to occur .\\
\textbf{Results:} The break appeared between the lines successfully.\\
\textbf{Errors:} None\\
\textbf{Solutions/Notes:}

\subsection{Objective: Insert a Table of Contents}
\textbf{Time Stamp:} 10:30pm 21 August\\
\textbf{Actions:} Insert \verb|\tableofcontents| at the beginning of the document after \verb|\maketitle|.\\
\textbf{Results:} The table of contents was created successfully.\\
\textbf{Errors:} None\\
\textbf{Solutions/Notes:}

\subsection{Objective: Insert a Cross-reference to Another Section}
\textbf{Time Stamp:} 11:00pm 21 August\\
\textbf{Actions:} With the hyperref package already in the preamble, insert \verb|\autoref{sec:nameforlink}| where you want the link to be, and \verb|\label{sec:nameforlink}| behind the section/subsection command of the section/subsection you want to link to. This command will make the link be 'section/subsection no'.\\
\textbf{Results:} The link was created successfully.\\
\textbf{Errors:} None\\
\textbf{Solutions/Notes:} This was a useful resource, \url{https://en.wikibooks.org/wiki/LaTeX/Labels_and_Cross-referencing#Sections} as recommended by Brian. This also works for images, if I switch 'sec' to 'fig'.

\subsection{Objective: Adding a Commit Description from An Overleaf Push}\label{sec:description}
\textbf{Time Stamp:} 1:05pm 23 August\\
\textbf{Actions:} In the Overleaf file, go to Menu and under Sync click GitHub. Type what I want the commit message to be on the first line, and then click enter two times. Type what I want the description to be and then click Commit. \\
\textbf{Results:} See Errors.\\
\textbf{Errors:} The push was not appearing in GitHub, the previous push was still being displayed as the latest commit.\\
\textbf{Solutions/Notes:} No changes had been made to the file on Overleaf so the push was not appearing. After a change was made to the Overleaf file, the same steps were followed and this worked successfully. 10:35pm 25 August - this method also works for commits from Sourcetree.

\subsection{Objective: Add a Comment}
\textbf{Time Stamp:} 10:15pm 25 August \\%This is a test comment
\textbf{Actions:} Use a percentage symbol and then type the comment. \\
\textbf{Results:} The comment was added successfully - it appeared in green and cannot be seen in the rendered pdf.\\
\textbf{Errors:} None.\\
\textbf{Solutions/Notes:} It was suggested in class if we find code from online sources to put a comment with where we found them for future reference. If I decide I would like the comments to appear in the margin of a document, refer to this source for how to do it,\\ \url{https://www.overleaf.com/blog/619-tip-of-the-week-add-inline-or-margin-comments-to-your-pdf}

\subsection{Objective: Delete a File}
\textbf{Time Stamp:} 10:20pm 25 August\\
\textbf{Actions:} When in the projects page of Overleaf, under Actions, click the Archive symbol (little box) of the file I want to delete. When the pop-up appears informing me I am about to archive the project, click the red Confirm button. Then in the sidebar, click Archived Projects and click the little trash can icon of the file(s) I want to delete under Actions. I can also select the check box of the file and click the red Delete Forever button.\\
\textbf{Results:} The file was deleted successfully.\\
\textbf{Errors:} None.\\
\textbf{Solutions/Notes:} In the Archived Projects page I can also restore the file by either selecting the check box and selecting the Restore button, or under Actions, selecting the Unarchive symbol (backward arrow/undo symbol).

\subsection{Objective: Add an Image}
\textbf{Time Stamp:} 1:25pm 29 August\\
\textbf{Actions:} In the OverLeaf file, I went to the upper left hand corner and clicked on the Upload button (little upwards facing arrow). In the pop-up, I clicked the green 'Select From Your Computer' button, found the image I wanted to use and pressed Open. The name of the image then appeared under the file name in the sidebar on the left. I then added \verb|\usepackage{graphicx}| to the preamble. To put the image in the document I entered \\
\verb|\begin{figure}[htp]| \\
\verb|\centering|\\
\verb|\includegraphics[width=4cm]{IMG_9495.PNG}|\\
\verb|\caption{Colour Maps of Disney Films in the 'Animated' App}|\\
\verb|\end{figure}|\\
\textbf{Results:} When typing in \verb|\includegraphics . . .| the line auto completed with the image file name. The image was successfully put into the document. I then played around with the image width until I was happy with the size.\\
\textbf{Errors:} None.\\
\textbf{Solutions/Notes:} If I want to sort my images if I have a large number in the file, I can click the New Folder button next to the Upload button in the top left hand corner of the Overleaf file. For more information, go to \url{https://www.overleaf.com/learn/how-to/Including_images_on_Overleaf}. Removing htp from the code put the image in the middle of my text.

\subsection{Objective: Add a Special Character (\textbar{})}
\textbf{Time Stamp:} 1:50pm 29 August\\
\textbf{Actions:} I wanted to add a \textbar{} in my text when completing the objectives and activities for The Unix Shell lesson, and I didn't initially realise when I was just entering \textbar{} it was appearing as |. So where I wanted them appear I inserted \verb|\textbar{}| in the text. \\
\textbf{Results:} When I recompiled the file, this inserted a \textbar{} successfully.\\
\textbf{Errors:} None.\\
\textbf{Solutions/Notes:} For ways to insert other special characters see, \url{https://en.wikibooks.org/wiki/LaTeX/Special_Characters}. An issue I had was that there was not space between \textbar{} and the next word. Adding \verb|{}| after \verb|\textbar| fixed this issue.

\pagebreak

\section{GitHub/Sourcetree}

\subsection{Objective: Commit a File}
\textbf{Time Stamp:} 9.00pm 12 August\\
\textbf{Actions:} Go into the repository into which I want to commit the file. Click upload files, and then drag and drop the file into the box. Enter information about the changes made in both description boxes. Click the green Commit changes button.\\
\textbf{Results:} File was successfully committed to GitHub.\\
\textbf{Errors:} None\\
\textbf{Solutions/Notes:} To commit a file using Sourcetree, save the file in the Hunt-Exercises folder on my computer. When I make a change in the document, I can go into Sourcetree and commit these changes. To commit from Overleaf, within the file, go to Menu and under Sync click GitHub, click Push Overleaf Changes to GitHub, add a commit message and click Commit. For adding a description with the commit message, see \autoref{sec:description}.

\subsection{Objective: Delete a File}
\textbf{Time Stamp:} 8:05pm 16 August\\
\textbf{Actions:} In the repository select the file I want to delete. Click on the trash can icon, and then click the red Commit button.\\
\textbf{Results:} File was successfully deleted.\\
\textbf{Errors:} None\\
\textbf{Solutions/Notes:}

\subsection{Objective: Change a Repository Name}
\textbf{Time Stamp:} 8:10pm 16 August\\
\textbf{Actions:} Within the repository, click settings. Under repository name, change the name to what I want it to be and click the Rename button.\\
\textbf{Results:} Name was successfully changed.\\
\textbf{Errors:} None\\
\textbf{Solutions/Notes:} I need to check whether when I push Overleaf changes to GitHub this still works. I did, and the changes were still pushed successfully (I think this was because even though the repository's name changed, the link remained the same).

\subsection{Objective: Create a Submodule}\label{sec:submodule}
\textbf{Time Stamp:} 8:15pm 12 August\\
I saw Brian in the consultation time before class, and he helped me set up a submodule with my learning journal Overleaf file. This was my attempt to set up a second submodule with my Second Scoping Exercise Overleaf file.\\
\textbf{Actions:} In the Overleaf file, go to Menu and under sync, click GitHub. When prompted click Create A GitHub Repository. Give the repository an appropriate name, give ownership to MQ-FOAR705 and click the green Create button. Go to GitHub and go to the newly created repository. Within the repository, click on the green Clone or Download button and copy the link. In Sourcetree, after clicking on Hunt-Learning Journal, go to Repository in the menu bar and go down to Add Submodule and click on it. Paste the previously copied link into Source Path/URL. Click on the ellipsis button for Local Relative Path (Make sure it opens the folder Hunt-Exercises). Create a new folder and name it appropriately and click OK. Then click Commit.\\
\textbf{Results:} The submodule was created in Sourcetree.\\
\textbf{Errors:} The submodule was not visible in GitHub.\\
\textbf{Solutions/Notes:} I had previously deleted two files in GitHub without having been into Sourcetree. The Pull icon was displaying a number two, and the Push icon was displaying a number one. I clicked Push and then OK and then Pull and then OK, and this addressed this issue.

\pagebreak

\section{Data Carpentry: Data Organisation in Spreadsheets for Social Scientists}

\subsection{Introduction}
\subsubsection*{Have I Used Spreadsheets in My Research?}
So far throughout my time at university, I have not used a spreadsheet in my research.
\subsubsection*{Have I Done Something that Made Me Frustrated or Sad?}
I have definitely done things that are frustrating, like not being able to find references I have used and accidentally leaving some out in my final submission. Also, formatting bibliographies and my research can be a frustrating process.

\subsection{Formatting Tables in Spreadsheets}
\subsubsection*{Activity 1: What is Wrong with the Messy Data and How Can It Be Fixed}
Mozambique 
\begin{itemize}
\item The absence of an observation is indicated in a number of different ways e.g. -99, -999 or cells are left blank. This could be addressed by choosing one and using it consistently
\item In the Dwelling table, there are a number of differences in spellings and identifications between the observations e.g. mabati\textunderscore sloping v mabatisloping. One should be chosen and consistently used.
\item In the Dwelling table, the yellow fill of the cell to indicate the inclusion of the barn will not be able to processed by the computer during analysis. Potentially the ownership of a barn could be included as a separate variable.
\item In the Livestock table, the livestock\textunderscore owned\textunderscore and\textunderscore numbers column contains two variables. To address this, it could be split into observations for each individual animal, as exemplified in the example.
\item In the water use column of the Plots table, the data switches between ‘no’ and ‘yes’ and Y and N. One should be chosen and consistently used.
\item Also in the water use column of the Plots table, there is an extra comment for key\textunderscore id 2. This should either be entirely removed, or separate columns set up for the different seasons.
\end{itemize}
Tanzania
\begin{itemize}
\item Similarly to the Mozambique Dwelling table, the use of an asterisk to indicate the inclusion of a cowshed is not an appropriate way to label the data. Cowsheds could be treated as a separate variable.
\item This is also the case in the Livestock table, where key\textunderscore id 3 has ‘yes/no*’ in the Look after cows column. Either yes or no should be chosen, or an observation omitted completely.
\item In the Livestock table, the data for key\textunderscore id 5 is recorded with yes and no, as opposed to the numerical data primarily used for the other responses. The yes/no entries should be changed into quantitative data.
\item There are a lot of blank cells in the Livestock table. A single indicator of a lack of response should be decided upon and used consistently.
\end{itemize}
Generally
\begin{itemize}
\item The data set for Tanzania is missing data on Plots.
\item Headers that have more than one word switch between having a space between them and an underscore.A single format for the headings should be consistently used.
\item The two tables recording Livestock are set up differently. Also, one table records responses for poultry as either yes or no, while the other records them numerically.These issues could be addressed by fixing the Mozambique Livestock take as previously discussed.
\item Across the two countries, the key\textunderscore ids consist of the numbers 1-10. When the two country’s data is merged, they would be understood to be the same people. To address this, Tanzania’s could be changed to 11-20.
\item Livestock data for key\textunderscore id 10 is missing from both tables.
\item All of the data should be placed into a single table
\end{itemize}
\subsubsection*{Activity 2: What types of metadata should be recorded with this project?}
\begin{itemize}
\item What does NULL indicate (was the question not asked, did the respondent not want to answer etc.) 
\item Definitions of the different variables. For example, what consistutes a mudduab wall type, or what is the basic definition of a room.
\item What were the exact questions asked in the survey to gather the responses.
\end{itemize}

\subsection{Formatting Problems}
\subsubsection*{Examples of Problem Data in my Discipline (Film Studies) }
\textbf{Problem 1: Data Spread Across Numerous Data Sets}\\
Data Sets: \url{https://www.imdb.com/interfaces/} and \url{https://archive.ics.uci.edu/ml/datasets/Movie}\\
The IMDb data is split into a number of datasets, each with different headers. For instance, the title.akas.tsv.gz data set has columns for the title, region for the version of the title and language of the title, whereas the title.basics.tsv.gz data set includes columns for the type/format of the title, primary title and original title.\\
This is similarly the case with the University of California Irvine Movie Data Set, which has various data sets for ‘People’, ‘Casts’, ‘Actors’, ‘Remakes’ and ‘Studios’. Additionally, in the various data sets, there are a number of columns that contain two variables. For example, in the Remakes data set, in the title and priortitle columns, there is a ’T:’ in front of every entry to indicate it is a film title.\\
Though discussed vary vaguely, data from the IMDb database was used for\\
\begin{itemize}
\item Sehwan Oh, Hyunmi Baek and JoongHo Ahn (2017) Predictive value of video-sharing behavior: sharing of movie trailers and box-office revenue, \textit{Internet Research}, 27:3, 691-708, DOI 10.1108/IntR-01-2016-0005.
\end{itemize}

\textbf{Problem 2: Null Values}\\
Data Sets: \url{http://www.cinemetrics.lv/database.php}\\
In the Cinemetrics Database, there are a number of cells left blank in a number of the entries and it is unclear why these cells have been left blank (e.g. is the data unavailable or is it not applicable?). Additionally, there are a number of repeated entries (such as for the film \textit{Insidious The Last Key}). With the database containing the average lengths of shots in films, a potential problem in the raw data is that it is susceptible to human error, as the length of a shot is manually measured by a viewer using a software that works like a stopwatch.\\
This misidentification of null values is also present in the previously mentioned University of California Irvine Movie Data Set. Incomplete data is identified in a variety of ways, including cells being left blank, and question marks.\\ 
The data from the Cinemetrics database appears in
\begin{itemize}
    \item James E. Cutting, Kaitlin L. Brunick and Jordan DeLong (2012) On Shot Lengths and Film Acts: A Revised View, \textit{Projections}, 6:1, 142-145, DOI: 10.3167/proj.2012.060106
    \item Lev Manovich (2013) Visualizing Vertov, \textit{Russian Journal of Communication},5:1, 44-55, DOI: 10.1080/19409419.2013.775546
\end{itemize}

\subsection{Dates as Data}
Please see GitHub for the Excel Sheets these activities were completed on.
\subsubsection*{Activity 1: Separating Dates Into Components}
\textbf{Objective: Add Columns}\\
\textbf{Time Stamp:} 12:46pm 18 August \\
\textbf{Actions:} Right click on the edge of the two columns between which I want to add a new column, and click Insert. Repeat this twice more, and label the columns 'Date', 'Month' and 'Year'.\\
\textbf{Results:} The three columns were created successfully.\\
\textbf{Errors:} None. \\
\textbf{Solutions/Notes:}\\
\vspace{5mm}
\textbf{Objective: Add Formulas to the Entire Columns}\label{sec:cellformat}\\
\textbf{Time Stamp:} 12:50pm 18 August \\
\textbf{Actions:} In the first blank cell of the day column (B2), type =DAY(A2) and click enter. Double click the bottom right hand corner of the cell to apply to remainder of the column. Repeat for the Month and Year columns by typing =MONTH(A2) into cell C2, and =YEAR(A2) into cell D2.\\
\textbf{Results:} This applied the formula successfully. \\
\textbf{Errors:} However, the data in the day, month and year columns appeared as dates. \\
\textbf{Solutions/Notes:} See the next Objective.\\
\vspace{5mm}
\textbf{Objective: Formatting Cells as a Number}\\
\textbf{Time Stamp:} 1:00pm 18 August \\
\textbf{Actions:} Highlight all the cells I want to format. Go to Format in the menu bar and click Cells . . . Under the Category menu click Number. Change the decimal place value to 0 and click OK.\\
\textbf{Results:} This successfully formatted the cells to numbers.\\
\textbf{Errors:} None.\\
\textbf{Solutions/Notes:}\\

\subsubsection*{Activity 2}\label{sec:formulacolumn}
\textbf{Objective: Including An Extra Date Without the Year to See the Default Year}\\
\textbf{Time Stamp:} 1:07pm 18 August\\
\textbf{Actions:} I typed 17/11 into the next blank cell in the interview\textunderscore date column.
\textbf{Results:} This was automatically formatted into a year (17/11/2019) in the interview\textunderscore date column, revealing the current year is the default year in Excel.\\
\textbf{Errors:} The day, month and year columns were not automatically populated.\\
\textbf{Solutions/Notes:} The way I initially applied the formulas to the remainder of the columns (double clicking the bottom right hand corner of the cell) didn't apply the formulas to any of the cells in the column beyond those which already contained data in the interview\textunderscore date column. I then tried alternate ways of applying the formulas to more cells, first by selecting a cell with the formula and dragging it down the column, and, then by copying a cell with the formula and pasting into a blank cell within the column without the formula. These both populated the corresponding cells with the new data I had entered. To test whether it would do this automatically, I deleted the new data (17/11/2019) from the interview\textunderscore date column. This produced 0, 1 and 1900 in the corresponding cells in the day, month and year columns respectively. I then re-entered 17/11 in the interview\textunderscore date column, and this automatically populated the columns with the data. 2019 appeared in the year column, further demonstrating how the current year is the deafult year used by Excel.\\

\subsection{Quality Assurance}
Please see GitHub for the Excel Sheets these activities were completed on.
\subsubsection*{Activity 1: Restricting Data to a Numeric Range}
\textbf{Objective: Follow the Instructions in the Lesson to Add a Restriction to the no\_membrs Column}\\
\textbf{Time Stamp:} 2:46pm 18 August\\
\textbf{Actions:} Select the entire no\_membrs column by clicking the top of the column (D). Go to the Data tab and select Data Validation (has a symbol with two boxes, a green tick and a red circle). A window appears, make sure the Settings tab is selected. Under Validation Criteria, from the Allow drop down menu, select Whole Number. Make sure under Data, 'between' is selected. Enter a minimum and maximum value, which for this example is 1 and 30 respectively. To create a specific message to appear when entering data, go to the Input Message tab. Enter a title (e.g. Invalid Data) and Input Message (e.g. Number of household members must be a whole number between 1 and 30.). If there is a tick in the box for 'Show input message when cell is selected', every time a cell in this column is selected, the typed message with appear. Select OK.\\
\textbf{Results:} This was successful as when 1.5 was attempted to be entered in the column, an Alert appeared that the value didn't match the data validation restrictions.\\
\textbf{Errors:} None. \\
\textbf{Solutions/Notes:} To customise the alert that appears when invalid data is entered, go to the Error Alert tab in Data Validation. I added a message with the title 'This Data Is Invalid' that says 'You entered a whole number that was not between 1 and 30.\\
\vspace{5mm}
\textbf{Adding a New Numeric Data Validation Rule of My Choosing}\\
Following the steps just outlined, I will attempt to add my own rule to another column with numeric data.\\
\textbf{Rule:} I added a restriction of whole numbers between 1 and 10 to the rooms column, as assessing the data in the table, it appears unlikely a house would have more than 10 rooms, and cannot have less than 1.\\
\textbf{Messages:} The input message I added was titled 'Invalid Data' and said 'The number of rooms must be a whole number between 1 and 10.'. I also added an error alert titled 'This Data Is Invalid', and with the message 'You entered a whole number that was not between 1 and 10.'.\\
\textbf{Result:} This was successful, as the input message appeared as intended, and when tested in the same way as before, the alert message also appeared as intended.

\subsubsection*{Activity 2: Restricting Data to Entries from a List}
\textbf{Objective:Follow the Instructions in the Lesson to Add a Restriction to the respondent\_wall\_type Column}\\ 
\textbf{Time Stamp:} 3:30pm 18 August\\
\textbf{Actions:} Select the entire respondent\_wall\_type column by clicking the top of the column (F). Go to the Data tab and select Data Validation. A window appears, make sure the Settings tab is selected. Under Validation Criteria, from the Allow drop down menu, select List. Under Source, enter all the options you want to allow to be entered into the column separating each with a comma (for this example: grass, muddaub, burntbricks, sunbricks, cement). As in Activity 1, add an appropriate input message (Invalid Data: Wall type must be either grass, muddaub, burntbricks, sunbricks or cement.) and error alert (This Data is Invalid: You entered a wall type that was not grass, muddaub, burntbricks, sunbricks or cement.).\\
\textbf{Results:} The data restriction was successfully added as the input message appeared as intended and when 'mud' was attempted to be entered into the column, the error alert appeared.\\
\textbf{Errors:} None.\\
\textbf{Solutions/Notes:} As the box, 'In-cell drop-down' was ticked in the Data Validation window in the Setting tab, when a cell in the column is selected, a down arrow appears on the right hand side of the cell, and when clicked down, a menu appears with the accepted data options. When one is selected, it is entered into the cell.\\
\vspace{5mm}
\textbf{Adding a New List Data Validation Rule of My Choosing}\\
Following the steps just outlined, I will attempt to add my own rule to another column with cateogrical data.\\
\textbf{Rule:} I added a list of restrictions to the months\_lack\_food column. The list included the abbreviations already used in the column (Jan, Feb, Mar, Apr, May, June, July, Aug, Sept, Oct, Nov, Dec). I chose this column to add a restriction to as months can be abbreviated in a number of different ways and it is important to ensure consistency.\\
\textbf{Messages:} The input message I added was titled 'Invalid Data' and said 'The month abbreviation must be Jan, Feb, Mar, Apr, May, June, July, Aug, Sept, Oct, Nov or Dec'. I also added an error alert titled 'This Data Is Invalid', and with the message 'You entered a month that was not Jan, Feb, Mar, Apr, May, June, July, Aug, Sept, Oct, Nov or Dec.'\\
\textbf{Result:} This was successful as the input message successfully appeared and when I tried enter 'Jan;Dece', I received the alert message I had entered. As this was such a long list and the cell can contain more than one month, I decided not to have a drop down menu in each individual cell.\\
\textbf{Error}: 9:00am 19 August, However I realised that entering the list of options for this column would only allow one month would be selected. 'Jan;Dece' was not a good way to test the success of this data validation rule because instead of the alert message presenting because 'Dece' was used instead of Dec, it was presenting because the two months was not a single option on the list. So, I then removed this data validation rule from the months\_lack\_food column, and applied a rule to the affect\_conflict column, only allowing never, once, more\_once, frequently,or NULL to be entered, and created appropriate alert and data entry messages. This worked as intended.\\

\subsection{Exporting Data}\label{sec:CSV}
\textbf{Objective: Save an Excel File in CSV format}\\ 
\textbf{Time Stamp:} 4:50pm 18 August\\
\textbf{Actions:} I will attempt to save my Dates as Data Activity Excel document in CSV Format. Inside the open document, in the menu bar go to File, and then select Save As . . . From the File Format drop down menu, select Comma-separated Values (.csv) and name the file appropriately. Select the correct folder to save it to and click Save. \\
\textbf{Results:} The file was exported successfully, and I uploaded it to my GitHub repository.\\
\textbf{Errors:} The export was not initially allowed because the spreadsheet had more than one sheet.\\
\textbf{Solutions/Notes:} I then deleted the MM/DD/YEAR sheet and tried again. This was successful and I uploaded it to my GitHub repository.\\

\pagebreak

\section{Software Carpentry: The Unix Shell}
\subsection{Introducing the Shell}
\textbf{Objective: Open the Unix Shell}\\ 
\textbf{Time Stamp:} 11:00am 25 August\\
\textbf{Actions:} Go to Spotlight Search in menu bar (top right hand corner of screen)(magnifying glass symbol). Type in 'Terminal' and open the top hit.\\
\textbf{Results:} It opened successfully with a prompt reading \verb|Emily$| appearing.\\
\textbf{Errors:} None.\\
\textbf{Solutions/Notes:}\\
\vspace{5mm}
\textbf{Objective: List the Current Directory}\\ 
\textbf{Time Stamp:} 11:05am 25 August\\
\textbf{Actions:} After \verb|Emily$| type in ls (stands for listing) and press enter. This is a command.\\
\textbf{Results:} The listings of the current directory appeared.\\
\textbf{Errors:} None.\\
\textbf{Solutions/Notes:}\\
\vspace{5mm}
\textbf{Objective: Clear the Window}\\ 
\textbf{Time Stamp:} 12:00pm 25 August\\
\textbf{Actions:} Hold down control and click the letter 'L'.\\
\textbf{Results:} The window was cleared successfully.\\
\textbf{Errors:} None.\\
\textbf{Solutions/Notes:}\\
\vspace{5mm}

\subsection{Navigating Files and Directories}
\textbf{Objective: Finding My Current Directory}\\ 
\textbf{Time Stamp:} 12:45pm 25 August\\
\textbf{Actions:} After \verb|Emily$| enter 'pwd' (print working directory) and press enter.\\
\textbf{Results:} The response was  '/Users/Emily'.\\
\textbf{Errors:} None.\\
\textbf{Solutions/Notes:} The first /, is referencing the root directory, in which the Users directory can be found (as explained in the lesson). When a slash appears inside a name, its just a separator. When I open a new command prompt, I will be in my home directory to start. Using the listings command, typing 'ls /' will give me all the directories in my root directory.\\
\vspace{5mm}
\textbf{Objective: Add a Marker to Directory Names}\\ 
\textbf{Time Stamp:} 1:00pm 25 August\\
\textbf{Actions:} Type 'ls -F' and press enter. This is known as a switch/option.\\
\textbf{Results:} A /, was put behind each of the directories.\\
\textbf{Errors:} None.\\
\textbf{Solutions/Notes:} A @ indicates a link and a * indicates an executable. \\
\vspace{5mm}
\textbf{Objective: Make A Directory Be Identified by a Colour}\\ 
\textbf{Time Stamp:} 1:05pm 25 August\\
\textbf{Actions:} Type 'ls -G' and press enter.\\
\textbf{Results:} The directories appeared in purple/blue. \\
\textbf{Errors:} None.\\
\textbf{Solutions/Notes:} This option can be combined with the one in the previous objective. Typing 'ls -G -F' both colours the directories and puts a slash behind them.\\
\vspace{5mm}
\textbf{Objective: Getting Help}\label{sec:help}\\ 
\textbf{Time Stamp:} 1:15pm 25 August\\
\textbf{Actions:} I typed 'ls --help' and pressed enter.\\
\textbf{Results:} \\
\textbf{Errors:} This produced the response 'ls: illegal option'\\
\textbf{Solutions/Notes:} I just tried typing 'help' and this produced information on how to use different commands. At the start of this is also different help commands for specific problems (e.g. use 'info bash' to find out more about the shell in general). Also, typing 'man ls' produces a lot of information, however, I couldn't enter a command after this and had to exit Terminal and start again. The lesson says to quit the man pages, press 'Q'. I can also find help online.\\
\vspace{5mm}
\textbf{Objective: Exploring More ls Flags (Activity)}\\ 
\textbf{Time Stamp:} 1:32pm 25 August\\
\textbf{Actions:} I typed 'ls -l' and pressed enter, and then tried 'ls -l -h' and pressed enter.\\
\textbf{Results:} 'ls 'l' produced extra information with the files/directories in the home directory, such as the date of creation, their owner, and what appears to be the owner's permissions (staff appeared next to Emily for all of them). 'ls -l -h' converted what appeared as just a number for 'ls -l' into a file size expressed in B/K (human readable).\\
\textbf{Errors:} \\
\textbf{Solutions/Notes:}\\
\vspace{5mm}
\textbf{Objective: Listing Recursively and By Time (Activity)}\\ 
\textbf{Time Stamp:} 1:45pm 25 August\\
\textbf{Actions:} I typed 'ls -t' and pressed enter. Then I tried 'ls -R' and pressed enter, and finally 'ls -R -t'.\\
\textbf{Results:} 'ls -t' ordered the files and directories in my home directory from most recently changed to least recently changed. 'ls -R' produced a really long list which the lesson informed me is the contents of all the directories. 'ls -R -t', as the lesson says, sorted the 'files/directories in each directory . . by time of last change'.\\
\textbf{Errors:} \\
\textbf{Solutions/Notes:}\\
\vspace{5mm}
\textbf{Objective: Seeing the Contents of a Different Directory}\\ 
\textbf{Time Stamp:} 2:40pm 25 August\\
\textbf{Actions:} I typed 'ls -F Desktop' to see the contents of my desktop. \\
\textbf{Results:} The response was a list of all the files and directories on my desktop, with all the directories indicated with a /. \\
\textbf{Errors:} \\
\textbf{Solutions/Notes:} To look at a sub-directory put a slash and then the name of that directory e.g to look at the data-shell directory on the desktop, type 'ls -F Desktop/data-shell' and press enter.\\
\vspace{5mm}
\textbf{Objective: Change the Working Directory Using a Relative Path}\\
\textbf{Time Stamp:} 2:50pm 25 August\\
\textbf{Actions:} This will use the 'cd' command (change directory). I typed 'cd Desktop' and pressed enter, typed 'cd data-shell' and pressed enter, and then typed 'cd data' and pressed enter.\\
\textbf{Results:} This change of working directory was successful, which I was able to identify in a number of ways. Firstly, there was no text response. Secondly, when the prompt appeared after each command, before \verb|Emily$| the name of the directory I had made the working directory appeared. Thirdly, when I entered 'pwd' after completing all the actions, /Users/Emily/Desktop/data-shell/data appeared. \\
\textbf{Errors:} \\
\textbf{Solutions/Notes:} This process can also be combined into one line, 'cd Desktop/data-shell/data'. To return to the home directory, just type 'cd' and press enter.\\
\vspace{5mm}
\textbf{Objective: Change the Working Directory Using an Absolute Path}\label{sec:absolute}\\ 
\textbf{Time Stamp:} 4:00pm 25 August\\
\textbf{Actions:} Beginning with /Users/Emily, type 'cd' and then the path to the directory I want to make the working one e.g. /Users/Emily/Desktop/data-shell \\
\textbf{Results:} When the example was used this was successful, as determined by using the same tests as for the previous objective.\\
\textbf{Errors:} None.\\
\textbf{Solutions/Notes:} This allows me to make this the working directory from anywhere, even when I am inside one of the subdirectories of the directory I want to be the working one. Instead of /Users/Emily/, I can just use ~ (stands for current user's home directory)(can only be at the front of a command). Another shortcut is also - for the previous directory. e.g. cd -\\
\vspace{5mm}
\textbf{Activity: Absolute vs Relative Paths}\\ 
Amanda could use these commands to navigate to her home directory
\begin{itemize}
    \item 2. cd /
    \item 5. cd ~
    \item 8. cd
\end{itemize}
Looking at the solutions, these are correct, but she could have also used 7. cd ~/data/.. (although this is unnessarily complicated) and 9. cd ..\\
\vspace{5mm}
\textbf{Activity: Relative Path Resolution}\\
The command will display 4. original/ pnas\textunderscore final/ pnas\textunderscore sub/. This is correct, as the .. is in references to a higher directory.\\
\vspace{5mm}
\textbf{Activity: Reading Comprehension}\\
I think 2. ls -r -F, will produce the output listed. Looking at the solution, this is correct, but 3. ls -r -F /Users/backup could also have been used.\\
\vspace{5mm}
\textbf{Objective: Tab Completion}\\ 
\textbf{Time Stamp:} 4:40pm 25 August\\
\textbf{Actions/Results:} Following the steps in the lesson, inside data-shell as the working directory, I typed 'ls nor' and then pressed tab. This completed to 'ls north-pacific-gyre/'. I then pressed tab again and this completed to 'ls north-pacific-gyre/2012-07-03/'. I then pressed tab again, and the reply was a list with all the files in this directory.\\
\textbf{Errors:} None.\\
\textbf{Solutions/Notes:}\\
\vspace{5mm}

\subsection{Working with Files and Directories}
\textbf{Objective: Creating A Directory}\\ 
\textbf{Time Stamp:} 7:15pm 25 August\\
\textbf{Actions:} I first made the directory I wanted to create the new directory in (data-shell) the working directory. I then typed 'mkdir thesis' (thesis is the name of the new directory). \\
\textbf{Results:} This was succesfful as there was no response, when I entered ls -F the new directory appeared followed by a slash and when I accessed the folder from GUI, the new thesis folder was there. \\
\textbf{Errors:} None. \\
\textbf{Solutions/Notes:} The lesson has notes for good file names including not using spaces and not starting names with a dash.\\
\vspace{5mm}
\textbf{Objective: Create A Text File}\\ 
\textbf{Time Stamp:} 7:20pm 25 August\\
\textbf{Actions/Results:} I first had to make the new thesis directory the working directory by entering 'cd thesis'. I then types 'nano draft.txt' and pressed enter. This was successful as it opened the nano text editor in the window. I then typed the same text that was suggested in the lesson and when finished pressed control 'O' (to save the file). A line appeared down the bottom saying 'File Name to Write: draft.txt.' I pressed enter. A message appeared down the bottom saying \verb|[ Wrote 1 line ]|. I then pressed control 'X' to exit. This was successful as when I entered 'ls', draft.txt appeared.\\
\textbf{Errors:} None. \\
\textbf{Solutions/Notes:} In nano there are a number of instructions down the bottom. \verb|^| means press the control key.\\
\vspace{5mm}
\textbf{Objective: Creating Files a Different Way (Activity)}\\ 
\textbf{Time Stamp:} 7:35pm 25 August\\
\textbf{Actions:} I typed 'touch my\textunderscore file.txt' and pressed enter.\\
\textbf{Results:} When I view the directory using Finder (GUI) the file did show up. When I entered 'ls -l' in Terminal, the file size is 0. I am unsure of what the touch command did or why this would be useful. \\
\textbf{Errors:} None (I think?)\\
\textbf{Solutions/Notes:} The solutions informed me 'some programs do not generate output files themselves, but instead require that empty files have already been generated. When the program is run, it searches for an existing file to populate with its output. The touch command allows you to efficiently generate a blank text file to be used by such programs'. This command and its use now makes sense to me.\\
\vspace{5mm}
\textbf{Objective: Renaming a File}\\ 
\textbf{Time Stamp:} 7:42pm 25 August\\
\textbf{Actions:} I first had to return to the data-shell directory by entering 'cd ~/Desktop/data-shell'. To rename draft.txt in the thesis directory to quotes.txt, enter 'mv thesis/draft.txt thesis/quotes.txt'. \\
\textbf{Results:} This was successful as entering 'ls thesis' produced the response my\textunderscore file.txt and quotes.txt \\
\textbf{Errors:} None. \\
\textbf{Solutions/Notes:} The lesson says to use mv -i or mv --interactive to get a confirmation message before overwriting, but neither of these worked when I tested them on the quote.txt file.\\
\vspace{5mm}
\textbf{Objective: Moving a Files}\\ 
\textbf{Time Stamp:} 8:00pm 25 August\\
\textbf{Actions:} The goal was to move quotes.txt into the current working directory (data-shell). I entered 'mv thesis/quotes.txt .' (the dot specifies the current directory; mv stands for move)\\
\textbf{Results:} This was successful as entering 'ls thesis' produced only my\textunderscore file.txt as a response, and as per the lesson instructions, entering 'ls quotes.txt' produced quotes.txt as a response. \\
\textbf{Errors:} None.\\
\textbf{Solutions/Notes:}\\
\vspace{5mm}
\textbf{Activity: Moving to the Current Folder}\\ 
Filling in the blanks: to move the files to the current folder, Jamie should enter \verb|$| mv \underline{analyzed}/sucrose.dat \underline{analyzed}/maltose.dat \underline{.} Checking the solution, I realised I should have had ../ before both 'analyzed'.\\
\vspace{5mm}
\textbf{Objective: Copying A File}\\ 
\textbf{Time Stamp:} 8:15pm 25 August\\
\textbf{Actions:} I entered 'cp quotes.txt thesis/quotations.txt' (cp stands for copy)\\
\textbf{Results:} This was successful, as confirmed by entering 'ls quotes.txt thesis/quotations.txt', and I also checked through the GUI (there was a file in the thesis folder called quotations.txt'\\
\textbf{Errors:} None.\\
\textbf{Solutions/Notes:} The second command demonstrates ls can be given multiple paths at once.\\
\vspace{5mm}
\textbf{Objective: Copying A Directory}\\ 
\textbf{Time Stamp:} 8:20pm 25 August\\
\textbf{Actions:} To copy the thesis directory and its entire contents (back it up) I entered 'cp -r thesis thesis\textunderscore backup'. \\
\textbf{Results:} This was successful as when I entered 'ls thesis thesis\textunderscore backup' the response indicated that both directories contained the files my\textunderscore file and quotations.txt. Accessing the data-shell folder through GUI also confirmed this. \\
\textbf{Errors:} None.\\
\textbf{Solutions/Notes:}\\
\vspace{5mm}
\textbf{Activity: Moving to the Current Folder}\\ 
To rename the file, you could use the second option 'mv statstics.txt statistics.txt'. The solution confirmed this was correct.\\
\vspace{5mm}
\textbf{Activity: Moving And Copying}\\ 
After completing this sequence, the ls command would produce option 2. recombine. This solution confirmed this was correct. While I got the right answer my reasoning was a little bit off, but the explanation provided with the solutions assisted the development of my understanding of what each of the commands meant. It also demonstrates to move a file to a directory, have 'mv filetobemoved.extension newdirectory/'\\
\vspace{5mm}
\textbf{Objective: Removing A File}\\ 
\textbf{Time Stamp:} 8:50pm 25 August\\
\textbf{Actions:} I typed 'rm quotes.txt' whule in the data-shell directory and pressed enter.\\
\textbf{Results:} The removal of this file was confirmed by entering 'ls quotes.txt' which produced the response 'ls: quotes.txt: No such file or directory'.\\
\textbf{Errors:} None.\\
\textbf{Solutions/Notes:}\\
\vspace{5mm}
\textbf{Activity: Using rm Safely}\\
I tried entering 'rm -i thesis\textunderscore backup/quotations.txt' While -i didn't work before, this time it did producing the response 'remove thesis\textunderscore backup/quotations.txt?'. This switch is important because as the lesson previously stated 'deleting is forever'. The solution said to use 'y' to confirm deletion, or 'n' to keep the file, but I also tried 'yes' and 'no', and these worked as well.\\
\vspace{5mm}
\textbf{Objective: Removing a Directory}\\ 
\textbf{Time Stamp:} 8:55pm 25 August\\
\textbf{Actions/Results:} As per the lesson instructions, I tried entering 'rm thesis' which produced 'rm: thesis: is a directory', proving rm does not work on directories. I then typed 'rm -r thesis' which deleted the directory successfully (as confirmed by entering 'ls' and through the GUI.\\ 
\textbf{Errors:} None.\\
\textbf{Solutions/Notes:} The lesson suggests using -i as these steps delete a directory and its entire contents, and \textit{should be used with great caution}\\
\vspace{5mm}
\textbf{Objective: Copy With Multiple File names (Activity)}\\ 
\textbf{Time Stamp:} 9:05pm 25 August\\
\textbf{Actions/Results:} As this activity was being completed in the data-shell/data directory I first entered 'cd data'. I also entered 'ls -F' to familiarise myself with the directory. Per the instructions, I then entered 'mkdir backup', followed by 'cp amino-acids.txt animals.txt backup/'. Entering ls backup confirmed a copy of each of the two files had been made in the backup directory. \\ I then tried the next commands, first entering 'ls -F' and then 'cp amino-acids.txt animals.txt morse.txt'. This produced a large chunk of text, which the solution informed me was because the last argument was not a directory.\\
\textbf{Errors:} None. \\
\textbf{Solutions/Notes:} I then tried entering 'cp planets.txt salmon.txt sunspot.txt backup/' to see if this would work with three files if the last argument was a directory. This worked successfully. I then tried deleting these three files using 'rm backup/planets.txt salmon.txt sunspot.txt'. This only deleted planets.txt. So I then tried 'rm backup/salmon.txt backup/sunspot.txt', and this successfully deleted the files.\\
\vspace{5mm}
\textbf{Activity: Wildcards - Listfilenames matching a pattern}\\
I first made the data-shell/molecules directory the working one by entering 'cd /Users/Emily/Desktop/data-shell/molecules'. The Wildcards information box is very useful, but to summarise, *.pdb would match (in this directory) every file that ends with .pdb, whereas p*.pdb would only match pentane.pdb and propane.pdb. As ? only matches one character, ?ethane.pdb would match methane.pdb, ???ane.pdb could also be used, which matches cubane.pdb ethane.pdb octane.pdb\\
The command which produce the output ethane.pdb methane.pdb in this directory, would be 1. ls *t*ane.pdb. Reviewing the solution the answer was actually 3. ls *t??ne.pdb. Upon reflection, this makes sense to me, as I didn't consider the files that weren't chosen when choosing my answer.\\
\vspace{5mm}
\textbf{Activity: More on Wildcards}\\
My responses to this activity were:\\
cp *dataset* backup/datasets\\
cp \underline{*}calibration\underline{.txt} backup/calibration\\
cp 2015-\underline{11-??-*}send\textunderscore to\textunderscore bob/all\textunderscore november\textunderscore files/\\
cp \underline{2015-??-23-*} send\textunderscore to\textunderscore bob/all\textunderscore datasets\textunderscore created\textunderscore on\textunderscore a\textunderscore 23rd/\\
The correct responses were:\\
cp *dataset* backup/datasets\\
cp \underline{*}calibration\underline{.txt} backup/calibration\\
cp 2015-\underline{11-*}send\textunderscore to\textunderscore bob/all\textunderscore november\textunderscore files/\\
cp \underline{*-23-dataset*}send\textunderscore to\textunderscore bob/all\textunderscore datasets\textunderscore created\textunderscore on\textunderscore a\textunderscore 23rd/\\
I understand where I went wrong, and how I could have simplified the commands.\\
\vspace{5mm}
\textbf{Activity: Organizing Directories and Files}\\
Jamie needs to enter 'mv *ose.dat/analyzed'. The answer was just 'mv *.dat/analyzed', but I'm really happy with how close I was able to get in writing this command without looking at any of my notes.\\
\vspace{5mm}
\textbf{Activity: Reproduce A Folder Structure}\\
I think the first two options would achieve the outlined objective. The issue with the third option is that no data directory is created before the raw and processed directories are attempted to be created so this wouldn't work. I think the problem with the last option is that the raw and processed directories have not been made subdirectories of the data directory. This was correct.\\

\subsection{Pipes and Filters}
\textbf{Objective: See the Word Count of Files in a Directory}\\ 
\textbf{Time Stamp:} 3:40pm 26 August\\
\textbf{Actions:} I first made the molecules directory the working one by entering 'cd /Users/Emily/Desktop/data-shell/molecules', and then used 'ls' to familiarise myself with its contents. I then typed 'wc *.pdb'. wc stands for word count, and this produces a list of the number of lines, words and characters (in that order). So, that specific command listed these for all the files in the directory which ended in .pdb. \\
\textbf{Results:} This produced the list as it appears in the lesson. \\
\textbf{Errors:} None. \\
\textbf{Solutions/Notes:} Typing 'wc -l *.pdb' just produces the number of lines. '-w' can be added for just the words, or '-c' for the number of characters. I also tested, and a combination of any two of these can also be used.\\
\vspace{5mm}
\textbf{Objective: Output Word Count Information to a File }\\ 
\textbf{Time Stamp:} 5:30pm 26 August\\
\textbf{Actions:} Type 'wc -l *.pdb \textgreater  lengths.txt' and press enter. This creates a file (lengths.txt) which contains the number of lines for each of the files indicated by *.pdb. The \textgreater  symbol is what 'redirect(s) the command's output to a file instead of printing it to the screen.\\
\textbf{Results:} This was successful, as confirmed by running 'ls lengths.txt' and through GUI.\\
\textbf{Errors:} None. \\
\textbf{Solutions/Notes:} Be careful as if there was already a file lengths.txt this would have overwritten it.\\
\vspace{5mm}
\textbf{Objective: View a File's Content in Terminal}\\ 
\textbf{Time Stamp:} 5:45pm 26 August\\
\textbf{Actions:} To see the file that was created in the previous objective onscreen, I typed 'cat lengths.txt' and pressed enter (cat stands for concatenate).\\
\textbf{Results:} This was successful as it printed the entire contents of the file onscreen.\\
\textbf{Errors:} None.\\
\textbf{Solutions/Notes:} This command prints the entire contents of the file to the screen. To view a file page by page, use the command 'less lengths.txt'. To go forward a page, press the space bar, and go back a page, press 'b'. To exit, press 'q'. Also, to get a specific number of lines enter, for example, 'head -n 1 filename.extension' (for the first line) or 'head -n 20 filename.extension' (for the first 20 lines).\\
\vspace{5mm}
\textbf{Activity: What Does Sort -n Do?}\\
Looking at the outputs, the addition of -n, has sorted the contents of the file numerically. The solution confirms that 'the -n option specifies a numerical rather than an alphanumerical sort'.\\
\vspace{5mm}
\textbf{Objective: Sorting the Contents of a File Numerically and Then Putting this in a New File}\\ 
\textbf{Time Stamp:} 6:50pm 26 August\\
\textbf{Actions/Results:} To sort the contents of the lengths.txt file from the previous objectives, I entered 'sort -n lengths.txt'.This sorted the contents of the file, from the the lowest number (least amount of lines) to the highest number (most amount of lines). To then put this information in a new file, the command 'sort -n lengths.txt \textgreater  sorted-lengths.txt' was entered. This was successful as when 'head -n 1 sorted-lengths.txt' (to view the first line), 9  methane.pdb appeared correctly.\\
\textbf{Errors:} None.\\
\textbf{Solutions/Notes:} As this has been sorted, entering 'head -n 1 sorted-lengths.txt' will give the file with the fewest lines. \\The lesson also notes that when directing the output of a sort to a new file, don't have the new file be the same name as the one the contents have been sorted from.\\ The sorting and word count commands are examples of a filter --\textgreater  they turn 'a stream of imput into a stream of output'.\\
\vspace{5mm}
\textbf{Activity: What Does \textgreater \textgreater  Mean?}\\
The lesson explains that the echo command prints strings e.g. entering 'echo The echo command prints text' produces the response 'The echo command prints text'. The instructions said to firstly enter 'echo hello \textgreater  testfile01.txt' and then 'echo hello \textgreater \textgreater  testfile02.txt', and then examine the output files (by entering 'cat testfile01.txt' and then testfile02.txt'). I did this, (both files contained 'hello') and then repeated the process. Now, testfile01.txt contained 'hello', while testfile02.txt contained 'hello hello'. This indicates that the \textgreater \textgreater  command adds to a file rather than overwrites it. This was confirmed by the answers.\\
\vspace{5mm}
\textbf{Activity: Appending Data}\\
The tail command functions the same as the head command, but prints lines from the end of a file. In light of this, the commands listed for this activity ('head -n 3 animals.txt \textgreater  animals-subset.txt', and 'tail -n 2 animals.txt \textgreater \textgreater  animals-subset.txt') would produce option 3. The first three lines and the last two lines of animals.txt. The answer confimed this is correct.\\
\vspace{5mm}
\textbf{Objective: Using a Pipe (Combining Commands)}\\ 
\textbf{Time Stamp:} 7:40pm 26 August\\
\textbf{Actions:} This pipe aims to combine the word count, sort and head commands used in the previous objectives. A pipe (\textbar{}) is used to separate the commands and tell the shell that 'we want to use the output of the command on the left as the input to the command on the right'. I entered 'wc -l *.pdb \textbar{}{} sort -n \textbar{} head -n 1'. This will the look at the number of lines in each file ending in .pdb in the molecules directory (the current working one), sort them numerically and print the first of the sorted files (i.e. the one with the fewest lines), all in one command.\\
\textbf{Results:} This was successful, as the response was 9 methane.pdb, like before when these commands were entered separately. \\
\textbf{Errors:} None.\\
\textbf{Solutions/Notes:}\\
\vspace{5mm}
\textbf{Activity: Piping Commands Together}\\
In the current directory, to find the three files which have the least number of lines I think option 4 ('wc -l * \textbar{} sort -n \textbar{} head -n 3') should be used. This was correct. When I tried this, this produced 1 testfile01.txt, 2. testfile02.txt and 7 lengths.txt.\\
\vspace{5mm}
\textbf{Activity: Pipe Reading Comprehension}\\
The pipeline provided (for the data-shell/data directory) was 'cat animals.txt \textbar{} head -n 5 \textbar{} tail -n 3 \textbar{} sort -r \textgreater  final.txt'. While I thought I understood what the first three pipes did, as well as the final one, I was unsure about what 'sort -r' did. The answers informed me it sorted the 3 lines from the previous pipe in reverse order. Consulting the solution, I also realised I didn't fully grasp that the 'tail -n 3' command would draw its input from the 'head -n 5' command, and not the entire animals.txt file. This now makes more sense to me.I then ran the entire command and it produced the same response as appears in the lesson.\\
\vspace{5mm}
\textbf{Activity: Pipe Construction}\\
The cut command can remove sections of each line (and is based on the expectation the lines will be separated into tab columns)(this is called a delimiter). In the command 'cut -d , -f 2 animals.txt', '-d' is used to indicate the comma as the delimiter character at where to cut, and the '-f' extract the second column. The uniq command deletes any consecutive double ups of lines in a file. The question said, what could I add to the 'cut -d , -f 2 animals.txt' command to find out what animals the file contains (without any duplicates in their names)? At first I was really unsure of the answer as I didn't really understand either what I was being asked to do, or exactly what the uniq command did (filters out adjacent matching lines in a file). When looking at the answer, I now understand that adding '\textbar{} sort \textbar{} uniq' to the end of the command, would allow the files to be sorted alphabetically, which would then allow the uniq command to identify and delete any double ups (as they would now be consecutive through the sorting). \\
\vspace{5mm}
\textbf{Activity: Which Pipe?}\\
When -c is added to the uniq command, a count is provided of the number of times a line occurs in its input. Based on this to produce a table that shows the total count of each animal in the file animals.txt, I think option 4 (cut -d, -f 2 animals.txt \textbar{} sort \textbar{} uniq -c)would be appropriate This is correct.\\
\vspace{5mm}
\textbf{Activity: Wildcard Expressions}\\
In writing wildcards, putting things in brackets like this [ ] means either or, e.g. *[AB].txt matches files ending in A.txt or B.txt. These are my answers to the questions, in the case I wasn't able to use [].
\begin{enumerate}
    \item You could use 'ls *A.txt' and 'ls *B.txt. This was correct.
    \item The small difference in the outputs would be that using the two described above would list all the files that end in A.txt and then all the files that end in B.txt, whereas *[AB].txt would list all the files ending in A.txt or B.txt in the order they are currently in. Looking at the solutions, this was partially right as the answer was 'the output from the new commands is separated because there are two commands'.
    \item I was unsure about the answer to this question. The solutions informed me my new expression would produce an error message where the original would not, 'When there are no files ending in A.txt, or there are no files ending in B.txt.' This makes sense to me now.
\end{enumerate}
\vspace{5mm}
\textbf{Activity: Removing Unneeded Files}\\
To delete the processed data files described, option 2 would be the best (rm *.txt). This is correct.\\

\subsection{Loops}
\textbf{Objective: Performing a Loop (Repeating a Command)}\\ 
\textbf{Time Stamp:} 8:15pm 27 August\\
\textbf{Actions:} The general form of a loop is: \\
'for thing in list\textunderscore of\textunderscore things\\
do\\
operation\textunderscore using \verb|$thing|\\
done'\\
So for this task, to get the second line in three files, I entered the commands\\
'for filename in basilisk.dat minotaur.dat unicorn.dat\\
do\\
head -n 2 \verb|$filename| \textbar{} tail -n 1\\
done\\
\textbf{Results:} When entering the second, third and fourth lines, the prompt changed to \textgreater . The lines appeared successfully. \\
\textbf{Errors:} None.\\
\textbf{Solutions/Notes:} The lesson suggests an indent for the third line to assist legibility. Each time the loop runs, it assigns a new file name to the filename variable (e.g. in this example, it assigns basilisk.dat then minotaur.dat then unicorn.dat) and runns the command. The variable name (\verb|$filename|) could be anything (e.g. temperature) but it is more beneficial to me to have it as something I can understand. Also, \verb|$filename| is the same as \verb|{$filename}| but different to \verb|${file}name|  \\
\vspace{5mm}
\textbf{Activity: Variables in Loops}\\
I ran the first code (each line of what I entered separated by ;)(for datafile in *.pdb; do; ls *.pdb; done). The output of this, was a list of the filenames repeated six times. In then ran the second code (for datafile in *.pdb; do; ls \verb|$datafile|; done). The output for this was a single list of the files. I think this is because the first command is essentially saying, for each of the files ending in .pdb, list all the files ending in .pdb, whereas the second one is just saying for all the files ending in .pdb list the file name. I think this is correct.\\
\vspace{5mm}
\textbf{Activity: Limiting Sets of Files}\\
The output of the loop listed would be option 4. Only cubane.pdb is listed, as this is indicated by c*. If *c* was used in replacement of c*, octane.pdb would be listed (as this matches *c*) as well as cubane.pdb (as the * before the c can also be 0 like in this instance). Both of these responses are correct as highlighted in the solution.\\
\vspace{5mm}
\textbf{Activity:Saving to a File in a Loop - Part One}\\
The effect of this loop would be 4. None of the above, as the output would be saved to a file called alkanes.pdb. This response was incorrect. Option 1 was the correct one (Prints cubane.pdb, ethane.pdb, methane.pdb, octane.pdb, pentane.pdb and propane.pdb, and the text from propane.pdb will be saved to a file called alkanes.pdb.). Upon reviewing the solution this now makes sense to me.\\
\vspace{5mm}
\textbf{Activity:Saving to a File in a Loop - Part Two}\\
While I originally thought option 1 was the correct output of the loop listed (All of the text from cubane.pdb, ethane.pdb, methane.pdb, octane.pdb, and pentane.pdb would be concatenated and saved to a file called all.pdb), I didn't realise that propane.pdb was not included in this option. Therefore option 3 is correct as it says all of the files (including propane.pdb) would be concatenated and saved to a file called all.pdb. This was correct.\\
\vspace{5mm}
\textbf{Objective: Run a More Complicated Loop}\\ 
\textbf{Time Stamp:} 9:40pm 27 August\\
\textbf{Actions:} The loop to be run will attempt to list the name of every file ending in .dat in the creatures directory, and then print lines 81-100 of these files. To do this I typed each of these lines and pressed enter:\\
for filename in *.dat\\
do]]
echo \verb|$filename|\\
head -n 100 \verb|$filename| \textbar{} tail -n 20\\
done\\
\textbf{Results:} This worked successfully, as the filename was listed and then the right amount of lines. \\
\textbf{Errors:} None. \\
\textbf{Solutions/Notes:} If the names of the files I am going to contain in a loop contain spaces, put the file in quote marks (e.g. in this objective if rather than *.dat, I had two files I would type them in as "red dragon.dat" and "purple unicorn.dat"). I would also have to put the loop variable in quotes (e.g. in this example, \verb|"$filename"| would appear in the third line.\\
\vspace{5mm}
\textbf{Objective: Make a Copies of Original Files Using a Loop}\\ 
\textbf{Time Stamp:} 10:00pm  27 August\\
\textbf{Actions:} The aim of this objective is to make a copy of each of the files in the creatures directory, with the name of the copy being original-originalfilename.dat (e.g. the copy of basilisk.dat will be original-basilisk.dat). This cannot simply be achieved be entering 'cp *dat original-*.dat' as cp has received more than two output nd expects the last to be a directory to copy the files to (which it isn't). Therefore, to achieve this aim this loop was entered\\
for filename in *dat\\
do\\
cp \verb|$filename| original-\verb|$filename|\\
done\\
\textbf{Results:} This successfully duplicated all the files with the intended name.\\
\textbf{Errors:} \\
\textbf{Solutions/Notes:} If I wanted to see what the commands in this loop would be without actually executing them, I could just add 'echo' to the beginning of the third line of the loop. This is a good way to check a loop before I run it.\\
\vspace{5mm}
\textbf{Objective: Moving Around the Lines in Terminal}\\ 
\textbf{Time Stamp:} 10:10pm  27 August\\
\textbf{Actions/Results:} At the beginning of a new prompt after running a loop I pressed the up arrow. This displayed the loop on a single line with each part separated by a semi-colon. This works for other commands too (just press the up arrow to copy the most recent one, or enter '!!' To move to the beginning of this line I pressed control 'A'. To move back to the end, I pressed control 'E'. \\
\textbf{Errors:} None. \\
\textbf{Solutions/Notes:}\\
\vspace{5mm}
\textbf{Objective: Repeat a Previous Command}\\ 
\textbf{Time Stamp:} 10:20pm  27 August\\
\textbf{Actions:} To look at the history of your commands type 'history' and press enter. You can also limit the history printed by typing 'history \textbar{} tail/head -n (no)'. I just tried entering 'history' and this printed all the commands I had entered since first using Terminal. To repeat a command, enter '!nocommand'. I entered '!179' which repeated the command in which I made the data directory my working one using an absolute path. \\
\textbf{Results:} This worked successfully and the data directory was made the working one. \\
\textbf{Errors:} \\
\textbf{Solutions/Notes:} I can also press control 'R' and then enter text, and Terminal will find the most recent command that matches what I have typed. Typing '!\verb|$| will get me the last word of the most recent command (e.g if the precedding commands ends in a file, I can type 'less !\verb|$|' to see the file.\\
\vspace{5mm}
\textbf{Activity: Doing a Dry Run}\\
At first, even after reading the solutions for this exercise I was really unsure. So I did as the solutions suggested and ran both of the loops. The first actually created the files (e.g. analyzed-cubane.pdb, analyzed-ethane etc.)(although I could not see the output as I did not have a program to open this type of file). Running the second one printed the steps of the loop in Terminal. I can now see how if the intention was to preview the commands the loop would run, the second loop (containing the line 'echo "analyze \verb|$file > analyzed-$file|"' would be more appropriate and the explanation in the solution makes more sense to me now.\\
\vspace{5mm}
\textbf{Activity: Nested Loops}\\
Again I was a little unsure, so I ran the code. I entered  as follows:\\
for species in cubane ethane methane\\
do\\
for temperature in 25 30 37 40\\
do\\
mkdir \verb|$species-$temperature|\\
done\\
done\\
This produced directories labelled e.g. cubane-25, cubane-30, cubane-37 and cubane-40, for cubane, ethane and methane. This now made more sense to me as I think the point of confusion was I thought this was extracting data related to temperature from the files, rather than just creating directories.\\
\vspace{5mm}

\subsection{Shell Scripts}
\textbf{Objective: Create and Run a Shell Script to Extract Specific Lines from a Specific File}\\
\textbf{Time Stamp:} 7:50pm 28 August\\
\textbf{Actions:} The shell created will attempt to select lines 11-15 of the file octane.pdb in the molecules directory which has been set as the working one. Firstly, I created a new file called middle.sh which will be the shell script by entering 'nano middle.sh'. This opens and creates the file in nano within Terminal. I then entered 'head -n 15 octane.pdb \textbar{} tail -n 5'. I then saved the file by pressing control 'O' and then enter, and exited nano by pressing control 'X'. I then entered 'ls' to confirm the file had been created. To run the shell script, I entered 'bash middle.sh'. \\
\textbf{Results:} This was successful as the lines 11-15 fro the file were printed to the screen, as they appear in the lesson. \\
\textbf{Errors:} None/\\
\textbf{Solutions/Notes:}\\
\vspace{5mm}
\textbf{Objective: Edit the Shell Script from the Previous Objective to Run on Arbitrary Files}\\ 
\textbf{Time Stamp:} 8:00pm 28 August\\
\textbf{Actions:} I opened the shell script from the previous objective by entering 'nano middle.sh'. I then replaced 'octane.pdb' in the file with \verb|"$1"| so it reads, 'head -n 15  \verb|"$1"| \textbar{} tail -n 5'. I then saved the file by pressing control 'O' and then enter, and exited nano by pressing control 'X'. The \verb|"$1"|, means 'the first filename (or other argument) on the command line'. I then entered 'bash middle.sh pentane.pdb' to run the shell on pentane.pdb. \\
\textbf{Results:} This worked successfully, printing the correct lines from the file (checked against the lesson).\\
\textbf{Errors:} None.\\
\textbf{Solutions/Notes:} The lesson notes the \verb|$1| is surrounded in double quotes for the same reason the loop variable was contained in quotations (in case the file name contains spaces).\\
\vspace{5mm}
\textbf{Objective: Edit the Shell Script from the Previous Objective to Allow the Range of Lines Printed to be Adjusted}\\ 
\textbf{Time Stamp:} 8:20pm 28 August\\
\textbf{Actions:} I again opened the middle.sh file in nano by entering 'nano middle.sh'. I changed 15 to \verb|$2| and 5 to \verb|$3|, so it read 'head -n \verb|"$2"| \verb|"$1"| \textbar{} tail -n \verb|"$3"|'. I then saved the file by pressing control 'O' and then enter, and exited nano by pressing control 'X'. I then entered 'bash middle.sh pentane.pdb 20 5' to print the lines 16-20 of the file pentant.pdb. \\
\textbf{Results:} This worked successfully, printing the lines as they appeared in the lesson. \\
\textbf{Errors:} None.\\
\textbf{Solutions/Notes:}\\
\vspace{5mm}
\textbf{Objective: Add a Comment to a Shell Script}\\ 
\textbf{Time Stamp:} 8:35pm 28 August\\
\textbf{Actions:} I opened the shell script from the previous objective, entering 'nano middle.sh'. At the top, I added a comment by starting the line with '\verb|#|'. As suggested in the lesson, I entered\\
\verb|#| Select lines from the middle of a file.\\
\verb|#| Usage: bash middle.sh filename end\textunderscore line num\textunderscore lines\\
I then saved this and exited nano.
\textbf{Results:} The addition of the comment was successful as when I ran the shell is still worked as intended, and when I opened it in nano again the comment appeared. \\
\textbf{Errors:} Comments are very useful in assisting other people and myself identify what a shell script does. It is useful to include both what a shell script does and how it should be used as in this example. However, everytime I change a shell script, I need to make sure I change the comment. \\
\textbf{Solutions/Notes:}\\
\vspace{5mm}
\textbf{Objective: Create a Run a Shell Script to Process Many Files in a Single Pipeline}\\ 
\textbf{Time Stamp:} 8:45pm 28 August\\
\textbf{Actions:} Instead of \verb|"$1"| etc. being used as in the previous objectives, \verb|"$@"| will be used as this means 'all of the command-line arguments to the shell script' (which allows the shell script to process many files in a single pipeline and files beyond the current directory). This script will sort filenames by their length. I began by opening and creating a new file by entering 'nano sorted.sh'. In nano, I then entered the description,\\
\verb|#| Sort filenames by their length.
\verb|#| Usage: bash sorted.sh one\textunderscore or\textunderscore more\textunderscore filenames.\\
Underneath this I then entered the command\\
wc -l \verb|"$@"| \textbar{} sort -n\\
I then saved this and exited nano. I then entered 'bash sorted.sh *.pdb ../creatures/*.dat' to run the shell script.
\textbf{Results:} This successfully sorted all the files ending in *pdb by their length.\\
\textbf{Errors:} None.\\
\textbf{Solutions/Notes:}\\
\vspace{5mm}
\textbf{Activity: List Unique Species} \\
After reading the instructions for this exercise, I really had no idea where to start and was really confused, and had to consult the solution. It read as follows:\\
\verb|#| Script to find unique species in csv files where species is the second data field\\
\verb|#| This script accepts any number of file names as command line arguments\\
\verb|#| Loop over all files\\
for file in \verb|$@|\\
do\\
echo "Unique species in \verb|$file|:"\\
\verb|#| Extract species names\\
cut -d , -f 2 \verb|$file|\textbar{} sort \textbar{} uniq\\
done\\
Reading this made sense to me, but I realised I didn't remember as much of the information from the loops lesson as I would have liked, so I will now go back and review my notes from that lesson.
\vspace{5mm}
\textbf{Objective: Save Previous Commands in a File to Use for a Shell Script}\\ 
\textbf{Time Stamp:} 9:10pm 28 August\\
\textbf{Actions:} The example the lesson provides for saving commands used to create a graph is 'history \textbar{} tail -n 5 \textgreater  redo-figure-3.sh'. I attempted saving previous commands to a file by entering 'history \textbar{} tail -n 5 \textgreater  testsave.sh' \\
\textbf{Results:} To test whether this worked, I entered 'nano testsave.sh'. My five previous commands appeared in the file successfully.\\
\textbf{Errors:} None.\\
\textbf{Solutions/Notes:} The lesson notes people usually run a command first to see if it works, and then save it to a shell script using the history command.\\
\vspace{5mm}
\textbf{Activity: Why Record Commands in the History Before Running Them?} \\
I think commands are saved to the command log before being run so that if any errors occur during the running of the command, the command can still be recovered. The solution confirmed this was right stating, 'if a command causes something to crash or hang, it might be useful to know what that command was, in order to investigate the problem'.\\
\vspace{5mm}
\textbf{Activity: Variables in Shell Scripts} \\
I think the correct answer is option 2. he first and the last line of each file ending in .pdb in the molecules directory. As the two ones would be in the place of the \verb|$2| and \verb|$3|, meaning the first and last lines of the files would be printed, and '*.pdb' means these would be printed for each file in the directory ending in .pdb. The solution confirms this to be correct.\\
\vspace{5mm}
\textbf{Activity: Finding the Longest File With a Given Extension} \\
text\\
\vspace{5mm}
\textbf{Activity: Script Reading Comprehension} \\
text\\
\vspace{5mm}
\textbf{Activity: Debugging Scripts} \\
When I ran the script in debug mode, it presented these lines for each of the files\\
bash goostats NENE01729A.txt\\ stats-NENE01729A.txt\\
+ for datafile in \verb|'"$@"'|\\
+ echo\\
As nothing is printed for echo, this must be the issue, reviewing the shell script \verb|$datfile| is in the place of \verb|$datafile|. The answers confirm this is correct.
\vspace{5mm}

\subsection{Finding Things}
\textbf{Objective:}\\ 
\textbf{Time Stamp:} pm  August\\
\textbf{Actions:} \\
\textbf{Results:} \\
\textbf{Errors:} \\
\textbf{Solutions/Notes:}\\
\vspace{5mm}
\textbf{Activity:} \\
text\\
\vspace{5mm}


\pagebreak

\section{Errors}
\subsection{Overleaf}

\subsubsection{Useful Errors}
\autoref{sec:underfull} Underfull /hbox (badness ****) in paragraph at lines **\\
\autoref{sec:logcommand} Log Command Has Changed
\autoref{sec:description} Commit From Overleaf not Appearing in GitHub

\subsubsection{Error: Overfull hbox (**pts too wide) in paragraph at lines **}
\textbf{Errors:} This error keeps appearing down the left hand side of the document where the lines are shown.\\
\textbf{Solutions/Notes:} I can find information on this from \url{https://www.Overleaf.com/learn/latex/Errors/Underfull_%5Chbox}
As this seems error seems to be indicating the text exceeds the margins of the page, I will try using \verb|\\| to put the text on a new line. 

\subsubsection{Error: Check that your \$'s match around math expressions. If they do, then you've probably used a symbol in normal text that needs to be in math mode. . .}
\textbf{Time Stamp:} 2:30pm 18 August\\
\textbf{Errors:} This kept appearing when I used an underscore in the text, and the error notification informed me that some symbols are used for mathematical calculations and have to be in math mode. I had previously just replaced underscores with dashes but wanted to address this issue.\\
\textbf{Solutions/Notes:} I used \verb|\textunderscore| to insert underscores where I needed them. However, when writing this journal entry where I had to include a dollar sign, I also discovered you can simply put a \verb|\| before the math symbol, or could use the same verb command I use when including code in the text.

\subsection{GitHub/Sourcetree}

\subsubsection{Useful Errors}
\autoref{sec:submodule} Submodule Not Visible in GitHub

\subsubsection{Error: Pushes from Overleaf Appearing in Connected Repository, but Not When it is Accessed Through Main Hunt-Exercises Repository}
\textbf{Time Stamp:} 10:00pm 16 August(Error)/10:00 am 18 August (Solution attempted)\\
Prior to attempting a solution, I asked advice from Brian on Slack.\\
\textbf{Actions:} Within Sourcetree, a Commit notification appeared. I clicked Commit, and within the description wrote, 'Commit changes from subdirectory into Hunt-Exercises', as well as the comment I put on the original push. I then clicked Commit. This then produced a one on the push symbol. I then completed the push and went into GitHub.\\
\textbf{Results:} This seemed to successfully make the changes appear in the repository.\\
\textbf{Notes:} I will now complete a new push with this added objective within the Overleaf file and follow these steps to see if it works. This worked successfully. Note however, prior to the first Commit notification appearing in Sourcetree, there is a Pull notification within the subdirectory that is being changed.

\subsection{Excel}
\subsubsection{Useful Errors}
\autoref{sec:cellformat} Cells in the Incorrect Format\\
\autoref{sec:formulacolumn} Cells Which Should Have a Formula Applied Not Automatically Populating\\
\autoref{sec:CSV} CSV Export Not Working\\

\subsection{The Unix Shell}
\subsubsection{Useful Errors}
\autoref{sec:help} Illegal Option/Getting Help\\

\subsubsection{Command Not Found}
\textbf{Time Stamp:} Numerous Occasions\\
\textbf{Action:} Check the spelling of the command. If this is not the issue, check online to see if the command is different.\\

\subsubsection{Error: Not Being Able to Move Up a Directory After Changing Working Directory (Using a Relative Path)}
\textbf{Time Stamp:} 3:20pm 25 August\\
\textbf{Actions:} I typed 'cd . .' and pressed enter. \\
\textbf{Results:} At first this didn't work and the response was 'command not found'. I then realised that there are not meant to be spaces between the dots and tried 'cd ..'. This worked and the prompt appeared with the parent (the directory containing) of the directory I had changed to the working directory. Also see Objective: Change the Working Directory Using an Absolute Path in \autoref{sec:absolute}\\
\textbf{Notes:} To be able to see the special directory (..) when running ls, type 'ls -F -a' (show all). In the response, the single dot is the current working directory. '.bash\textunderscore profile' is the file containing shell configuration settings and other files may have dots in front of them (see Navigating Files and Directories lesson for more details). To summarise .. means 'the directory above the current one' and . means 'the current directory'.

\subsubsection{Word Count Command Not Working}
\textbf{Time Stamp:} N/A\\
\textbf{Action:} As noted in the lesson, if you type 'wc' and click enter, you haven't added a file name so Terminal waits for you to provide input for it to process. To exit this, press control and 'c'.\\

\end{FlushLeft}


\end{document}
